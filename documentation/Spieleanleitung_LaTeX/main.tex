\documentclass[a4paper,12pt]{article}
\usepackage[utf8]{inputenc}
\usepackage[ngerman]{babel}
\usepackage{graphicx}
\usepackage{hyperref}

\title{Spielanleitung: Where tf am I?}
\author{Niklas Fichtner, Hikmet Gözaydin, Oskar Pokorski}

\begin{document}

\maketitle

\begin{center}
\textbf{Version 1.0}
\end{center}

\tableofcontents
\newpage

\section*{Einleitung}
Diese Spielanleitung dient dazu, neue Spieler schnell in die Welt von \textit{Where tf am I} einzuführen. Sie beschreibt die Installation, Bedienung und das Ziel des Spiels sowie wichtige Funktionen wie Pausieren und das Handling von Level-Ergebnissen.

\newpage

\section{Installation}
\label{sec:installation}

Um \textit{Where tf am I} zu spielen, folge diesen Schritten:

\begin{itemize}
    \item Klone das GitHub-Repository des Projekts.
    \item Starte die ausführbare Datei (EXE), um das Spiel zu installieren und zu starten.
\end{itemize}

Das Spiel ist derzeit nur für PC verfügbar.

\newpage

\section{Menü}
\label{sec:menu}

Nach dem Start gelangst du ins Hauptmenü mit folgenden Optionen:

\begin{itemize}
    \item \textbf{Start}: Beginne das Spiel mit Level 1.
    \item \textbf{Choose Level}: Wähle ein freigeschaltetes Level aus.
    \item \textbf{Options}: Passe Einstellungen wie Sound an.
    \item \textbf{Quit}: Beende das Spiel.
\end{itemize}

\newpage

\section{Spielbedienung}
\label{sec:spielbedienung}

Die Steuerung ist einfach und wird in Level 1 eingeführt:

\begin{itemize}
    \item \textbf{Bewegung}: Nutze W (vorwärts), A (links), S (rückwärts), D (rechts).
    \item \textbf{Springen}: Drücke die Leertaste.
    \item \textbf{Schießen}: Klicke mit der linken Maustaste und richte den Mauszeiger auf Gegner, um einen Magieangriff auszuführen.
    \item \textbf{Kamera}: Die Kamera ist fest; der Mauszeiger dient als Zielhilfe.
    \item \textbf{Pausieren}: Drücke Esc, um das Pausemenü zu öffnen.
\end{itemize}

\newpage

\section{Ziel des Spiels}
\label{sec:ziel}

Das Ziel von \textit{Where tf am I?} ist es, jedes Level zu durchqueren, Gegner zu besiegen oder zu umgehen, und das Portal am Ende zu erreichen, ohne dass die Lebenspunkte auf null sinken.

\newpage

\section{Statusanzeigen: HP und Mana}
\label{sec:statusanzeigen}

Links oben im Bildschirm siehst du deine Statusanzeigen:
\begin{itemize}
    \item \textbf{HP (Lebenspunkte)}: Zu Erkennen an dem roten Balken. Zeigt deine verbleibenden Lebenspunkte an. Du startest mit 100 HP und verlierst HP, wenn dich Gegner treffen. Sinkt dein HP-Wert auf 0, verlierst du das Level.
    \item \textbf{Mana}: Zu Erkennen an dem blauen Balken. Zeigt deine verfügbare Mana-Menge an, die für Magieangriffe benötigt wird. Die Mana Punkte regenerieren sich mit der Zeit selbstständig.
\end{itemize}

\newpage

\section{Gegner und ihre Eigenschaften}
\label{sec:gegner}

Im Spiel triffst du auf drei Gegnerarten, die sich in ihren Eigenschaften unterscheiden:

\begin{itemize}
    \item \textbf{Magier (normal)}
        \begin{itemize}
            \item HP: 60
            \item Laufgeschwindigkeit: 200 (Einheiten pro Sekunde)
            \item Angriffsgeschwindigkeit: 3 Sekunden Abklingzeit
            \item Schaden pro Hit: 10 HP
        \end{itemize}
    \item \textbf{Schnellschuss-Magier}
        \begin{itemize}
            \item HP: 30
            \item Laufgeschwindigkeit: 50
            \item Angriffsgeschwindigkeit: 0,1 Sekunden Abklingzeit
            \item Schaden pro Hit: 5 HP
        \end{itemize}
    \item \textbf{Tank-Magier}
        \begin{itemize}
            \item HP: 200
            \item Laufgeschwindigkeit: 400
            \item Angriffsgeschwindigkeit: 6 Sekunden Abklingzeit
            \item Schaden pro Hit: 20 HP
        \end{itemize}
\end{itemize}

Die Gegner verfolgen dich, sobald du ihnen Nahe kommst und greifen an. Ihre Angriffe richten sich immer auf deine Position.

\newpage

\section{Level pausieren}
\label{sec:pause}

Um das Spiel zu pausieren, drücke die Esc-Taste. Im Pausemenü kannst du:
\begin{itemize}
    \item Zurück zum Hauptmenü gehen (falls implementiert).
    \item Das Spiel fortsetzen (Esc erneut drücken).
\end{itemize}

\newpage

\section{Level geschafft}
\label{sec:levelgeschafft}

Wenn du ein Level erfolgreich abschließt, erscheint ein Bildschirm mit:
\begin{itemize}
    \item Das nächste Level auswählen.
    \item Zum Hauptmenü zurückkehren.
\end{itemize}

Klicke auf die entsprechende Schaltfläche, um fortzufahren.

\newpage

\section{Level verloren}
\label{sec:levelfehler}

Du verlierst, wenn deine HP auf null fallen (z. B. durch Treffer von Gegnern) oder wenn du aus der Map herausfliegst. Dann siehst du einen Bildschirm mit:
\begin{itemize}
    \item "You died"
    \item Level neustarten.
    \item Zum Hauptmenü zurückkehren.
\end{itemize}

Wähle die gewünschte Aktion, um fortzufahren.

\end{document}