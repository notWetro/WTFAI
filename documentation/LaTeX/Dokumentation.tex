\documentclass[oneside]{ausarbeitung}
\bibliography{latexlit}


% ----------------------------------------------------------------------

\begin{document}

%--- Sprachauswahl
% Erlaubte Werte:
%   \selectlanguage{english}
%   \selectlanguage{ngerman}
\selectlanguage{ngerman}

%--- Art der Arbeit
% Erlaubte Werte:
%   \Praxissemesterbericht
%   \Projektbericht
%   \Bachelorarbeit
%   \Seminararbeit
%   \Masterarbeit

\Projektbericht

%--- Studiengang:
% Erlaubte Werte:
%   \Informatik
%   \Elektronik
%   \DataScience
\Informatik

\title{Where tf am I?}

\author{Niklas Fichtner; Hikmet Gözaydin; Oskar Pokorski}
\matrikelnr{85822; 85987; 81749}

%--- Ist der Erstbetreuer (\examinerA) an der Hochschule ein Professor?
% Erlaubte Werte:
%   \examinerIsAProfessortrue   % Ja
%   \examinerIsAProfessorfalse  % Nein
\examinerIsAProfessortrue   % Ja

%--- Betreuer
\examinerA{Stefan Wehrenberg}
%\examinerB{Prof.~Dr.~Ulrich~Klauck}

%--- Einreichungsdatum
\date{20. Juli 2025}

%--- Angaben zur Firma
% Auskommentieren, wenn die Arbeit nicht bei einer ext. Firma gemacht wurde.
%\companyname{Beispielfirma}
%\industrialsector{Beispielbranche}
%\department{Beispielabteilung}
%\companystreet{Beispielstr. 1}
%\companycity{12345 Musterstadt}

%--- Angaben zum Betreuer bei dieser Firma
%\advisorname{Name des Betreuers}
%\advisorphone{(01234) 567-890}
%\advisoremail{name@company.xxx}

%--- Titelseite Anzeigen
\maketitle
\cleardoublepage

%---
\pagenumbering{roman}
\setcounter{page}{1}

%--- Firmendaten Anzeigen
% Auskommentieren, wenn die Arbeit nicht bei einer ext. Firma gemacht wurde.
%\makeworkplace
%\cleardoublepage

%--- Eidesstattliche Erklärung anzeigen
\makeaffirmation
\cleardoublepage

%--- Sperrvermerk (Funktioniert nur bei externen Bachelor- oder Masterarbeiten.)
\makeconfidentialclause
\cleardoublepage

%---
\begin{abstract}
  Ziel der Kurzfassung ist es, einen (eiligen) Leser zu informieren, so 
  dass dieser entscheiden kann, ob der Bericht für ihn hilfreich ist oder 
  nicht (neudeutsch: Management Summary). Die Kurzfassung gibt daher eine 
  kurze Darstellung

  \begin{itemize}
    \item des in der Arbeit angegangenen Problems
    \item der verwendeten Methode(n)
    \item des in der Arbeit erzielten Fortschritts.
  \end{itemize}

  Dabei sollte nicht auf die Struktur der Arbeit eingegangen werden, also 
  Kapitel~\ref{cha:grundlagen} etc. denn die Kurzfassung soll ja gerade 
  das Wichtigste der Arbeit vermitteln, ohne dass diese gelesen werden muss.
  Eine Kapitelbezogene Darstellung sollte sich in Kapitel~%
  \ref{cha:einleitung} unter Vorgehen befinden.

  Länge: Maximal 1 Seite.
\end{abstract}
%-----------------------------------------------------------------------
\cleardoublepage
\tableofcontents

%---
\listoffigures

%---
\listoftables

%---
\lstlistoflistings

%---
\listofabbreviations
\begin{acronym}[Bsp.]  % Längstes Kürzel in der nachfolgenden
                       % Liste um die Breite der Spalte für die
                       % Abkürzungen zu bestimmen.

%% Eintrag: \acro{Referenzname}[Kürzel]{Langform}
%% Im Text wird die Abkürzung dann mit \ac{Referenzname} benutzt.
\acro{rup}[RUP]{Rational Unified Process}
%\acro{bsp}[Bsp.]{Beispiel}
\end{acronym}
%---


\cleardoublepage
\pagenumbering{arabic}
\setcounter{page}{1}

% ----------------------------------------------------------------------
\chapter{Einleitung}
\label{cha:einleitung}

H - Spielname, Genre, Kurzbeschreibung, Zielgruppe, Plattform...


Die Einleitung dient dazu, beim Leser Interesse für die Inhalte 
Praxissemesterberichts zu wecken, die behandelten Probleme aufzuzeigen 
und die zu ihrer Lösung entwickelten Konzepte zu beschreiben.

\section{Motivation}
\label{sec:motivation}

%---
\chapter{Gameplay und Mechaniken}
\lable{cha:gameplayundmechaniken}

\section{Spielmechaniken}
\lable{sec:spielmechaniken}

Da unser Speil Lebelbassiert ist, ist dies auch die Core Gameplay Loop. Somit ist die Hauptspielmechanik ins Portal am Ende vom Level zu kommen.
Dabei muss man die Gegner besieten oder ihnen ausweichen und darauf achten, dass seine Lebenspunkte nicht auf Null fallen. 


\section{Progression und Leveldesign}
\lable{sec:progressionundleveldesign}

O   - Progession durch Level abschluss
O   - Einleitende worte

\subsection{Level 1}
\lable{sub:level1}

O  - Level 1
   - Kurz beschreiben und Screenshot


\subsection{Level 2}
\lable{sub:level2}

N  - Level 2
   - Kurz beschreiben und Screenshot
Level 2 ist darauf ausgelegt, dass der Spieler die Bewegungs- und Sprungmechaniken besser kennenlernt und den anderen Gegnerarten zu begegnen. 
Dementsprechend kämpft mann zuerst gegen die "Tank"-Gegner und anschließend gegen den Schnellschussmagier. 
Anschließend folgt ein kleines Jump'n'Run an dessen Ende man das Portal vorfinden und zum nächsten Level forranschreiten kann.


\subsection{Level 3}
\lable{sub:level3}

H  - Level 3
   - Kurz beschreiben und Screenshot


\subsection{Level 4}
\lable{sub:level4}

O  - Level 4
   - Kurz beschreiben und Screenshot


\subsection{Level 5}
\lable{sub:level5}

N  - Level 5
   - Kurz beschreiben und Screenshot
Wenn man Level 5 betritt erscheint es einem Überraschend dunkel, im vergleich zu den anderen Leveln davor. 
Das liegt daran, das dieses Level Labyrint-Adventure-Setting hat und man mit seinem Magieschuss sich den Weg leuchten soll.
Es gibt mehrere Wege, von denen nur einer zum Portal führt. 

Dabei sollte man als Spieler nur beachten, das man nicht ins "Void" schiessen kann. 
Also sollte man ab besten in Fußnähe das Charakters klicken um zu schiessen.


\section{Steuerung und Eingaben}
\lable{sec:steuerungundeingaben}

Die Steuerung wird am Anfang von Level 1 erklärt und ist generell auch sehr intuitiv, da sie dem bekannten "Standart" entspricht. 
Also "wasd" zu laufen, Leertaste zum Springen und Linksklick für die Primäre Aktion, welche in unserem Fall das Abfeuern eines Magiegeschisses ist. 
Um die Richtung zu bestimmen, in die man schiessen möchte bewegt man einfach den Mauszeiger in die Entsprechende Richtung. 

Die Kameraperspektive ist fest, weswegen auch der Mauszeiger eingeblendet ist und man mit diesem Ziehlt, da man die Kamera nicht "Drehen" kann, 
wie es bei Spielen der Third-Person Perspektive gängig ist. 

Über die "Esc"-Tast kann man wie gewohnt das Pausemenu öffnen. 


\section{Schwierigkeitskurve und Balance}
\lable{sec:schwierigkeitskurveundbalance}

N - Schwierigkeitskurve & Balance
Das Spiel fängt von der Schwierigkeit her sehr leicht an, dass man Zeit hat die Steuerung und die Gamephysics kennen zu lernen. 
In Level 1 und 2 lernt man alle Gegnerarten kennen und wird mit dem Movement vertraut gemacht. 
In den darauf Folgenen Leveln steigt die Schwierigkeit und man begegnet mehr gegnern oder hat längere Jump'n'Run parkoure zu beweltigen. 
Dadurch bleibt das Interesse des Spielers erhalten, da eine gewisse Herrausforderung vorhanden ist. 
Die Level sind so abgestimmt, dass man nicht jedes beim ersten Versuch schaft aber auch nicht zu lange braucht um das Level zu schaffen. 


%---
\chapter{Story und Narrative}
\lable{cha:stroyundnarrative}

\section{Hintergrundgeschichte}
\lable{sec:hintergrundgeschichte}

H - Hintergrundgeschichte


H - (Erzählweise)


\section{Charaktere und Motivation}
\lable{sec:charaktereundmotivation}

H - Charaktere & Motivation


%---
\chapter{Grafik und Sound}
\lable{cha:grafikundsound}

\section{Visueller Stil und Inspirationsquellen}
\lable{sec:visuellerstilundinspirationsquellen}

O - Visueller Stil & Inspirationsquellen
O   - Realistisch-Gezeichnet oder so


\section{Farbschema und UI-Design}
\lable{sec:farbschemaundui-design}

O - Farbschema & UI-Design
O   - SchlichtesUI Design


\section{Sounddesign und Musik}
\lable{sec:sounddesignundmusik}

H - Sounddesign & Musik


%---
\chapter{KI}
\lable{cha:ki}

\section{Feindverhalten und KI-Routinen}
\lable{sec:feindverhaltenundki-routinen}

N - Feindverhalten & KI-Routinen


\section{NPC-Interaktionen}
\lable{sec:npc-interaktionen}

N - NPC-Interaktionen


\section{Gegnerarten}
\lable{sec:gegnerarten}

N - Gegnerarten


%---
\chapter{Technische Umsetzung}
\lable{cha:technischeumsetzung}

\section{Game Engine und Tools}
\lable{sec:gameengineundtools}

N - Game Engine & Tools


\section{Code-Architektur und Struktur}
\lable{sec:code-architekturundstruktur}

N - Code-Architektur & Struktur


%---
\chapter{Entwicklungsprozess}
\lable{cha:entwicklungsprozess}

\section{Projektplanung}
\lable{sec:projektplanung}

H - Projektplanung


\section{Meilensteine}
\lable{sec:meilensteine}

H - Meilensteine


%---

\chapter{Herausforderungen und Lösungsansätze}
\lable{cha:herausforderungenundlösungsansätze}

\section{Einarbeitung}
\lable{sec:einarbeitung}

O - Einarbeitung


\section{Technische Probleme}
\lable{sec:technischeprobleme}

  - Technische Probleme
N   - Synk via GitHub
N   - Build nach dem pullen


\section{Designentscheidungen und Änderungen}
\lable{sec:designentscheidungenundänderungen}

  - Designentscheidungen & Änderungen


%---
\chapter{Fazit}
\lable{cha:fazit}

\section{Aufgetretene Probleme}
\lable{sec:aufgetreteneprobleme}

  - Aufgetretene Probleme


\section{Erreichte Ziele}
\lable{sec:erreichteziele}

  - Erreichte Ziele


\section{Weiterentwicklungsmöglichkeiten}
\lable{sec:weiterentwicklungsmöglichkeiten}

  - Weiterentwicklungsmöglichkeiten


%---
\chapter{Quellen}
\lable{cha:quellen}

- Quellen (Tutorials, Assets, etc.)












%---Noch von alter Vorlage
\chapter{Implementierung}
\label{cha:implementierung}

In diesem Kapitel wird die konkrete Implementierung des im Kapitel
\ref{cha:loesungskonzept} entwickelten Lösungskonzepts beschrieben.
Hierbei wird auf die konkret verwendeten Entwicklungswerkzeuge etc. 
Bezug genommen.

Bei Software-Projekten besteht dieses Kapitel typischerweise aus den 
Phasen Implementierung \& Test im \ac{rup}.

Zum Beispiel kann man hier auch ein kleines Listing einfügen (Siehe \ref{lst:test}).

\begin{lstlisting}[language=c,%
                   caption={Überschrift des Quelltexts},label=lst:test]
#include<stdio.h>

int main() {
    // Kommentar
    int answer = 20 << 1;
    answer += 2;
    printf("Hallöchen Welt!\n");
    printf("Die Antwort ist: %d\n", answer);
    return 0;
}
\end{lstlisting}

Manchmal hilft auch eine kleine Tabelle:

\begin{table}[htbp]
\centering
\begin{tabular}{|l|r|}
\hline
\textbf{Messwert a} & \textbf{Messwert b} \\ \hline
9 & 5 \\ \hline
1 & 4 \\ \hline
1 & 3 \\ \hline
\end{tabular}
\caption{Überschrift der Tabelle}
\label{tab:my-table}
\end{table}

Details siehe Tabelle~\ref{tab:my-table}.


  \begin{itemize}
    \item des in der Arbeit angegangenen Problems
    \item der verwendeten Methode(n)
    \item des in der Arbeit erzielten Fortschritts.
  \end{itemize}

%-----------------------------------------------------------------------
\appendix

%---
\printbibliography[heading=bibintoc]

%---
\chapter{Anhang A}

\Blindtext

%---
\chapter{Anhang B}

\Blindtext

\end{document}
